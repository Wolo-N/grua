\documentclass[10pt,a4paper]{article}
\usepackage[spanish]{babel}
\usepackage[utf8]{inputenc}
\usepackage{amsmath}
\usepackage{amssymb}
\usepackage{graphicx}
\usepackage{booktabs}
\usepackage{geometry}
\usepackage{float}
\usepackage{multicol}
\usepackage{hyperref}

\geometry{margin=1.5cm}
\setlength{\parskip}{2pt}
\setlength{\columnsep}{12pt}

\title{\vspace{-1cm}\textbf{Trabajo Práctico 2: Análisis Estructural de Grúa mediante Elementos Finitos}}
\author{Ecuaciones Diferenciales al Modelado}
\date{\today}

\begin{document}

\maketitle
\vspace{-0.7cm}

\begin{abstract}
\small
Este trabajo presenta el análisis estructural de una grúa de 30 metros mediante elementos finitos. Se desarrollaron dos implementaciones: \textbf{GRUITA 2} con geometría fija para análisis rápido, y \textbf{GRUITA 3} con optimización automática de topología y dimensiones mediante algoritmos evolutivos. El objetivo es minimizar el costo manteniendo factores de seguridad FS $\geq$ 2.0 bajo cargas de hasta 40 kN.
\end{abstract}

\section{Hipótesis y Marco Teórico}

\subsection{Ecuaciones Fundamentales}

El método de elementos finitos para estructuras reticuladas se basa en:

\textbf{Matriz de rigidez elemental:}
\begin{equation}
\mathbf{K}_e = \frac{EA}{L} \begin{bmatrix}
c^2 & cs & -c^2 & -cs \\
cs & s^2 & -cs & -s^2 \\
-c^2 & -cs & c^2 & cs \\
-cs & -s^2 & cs & s^2
\end{bmatrix}
\end{equation}

donde $E = 200$ GPa, $A$ = área, $L$ = longitud, $c = \cos\theta$, $s = \sin\theta$.

\textbf{Sistema global:}
\begin{equation}
\mathbf{K}\mathbf{u} = \mathbf{F}
\end{equation}

\textbf{Esfuerzo axial:}
\begin{equation}
\sigma = E \frac{\Delta L}{L}, \quad F_{axial} = \sigma \cdot A
\end{equation}

\textbf{Pandeo (Euler):}
\begin{equation}
P_{cr} = \frac{\pi^2 EI}{L^2}, \quad FS_{pandeo} = \frac{P_{cr}}{|P_{max}|}
\end{equation}

\textbf{Secciones tubulares:}
\begin{equation}
A = \pi(r_{ext}^2 - r_{int}^2), \quad I = \frac{\pi}{4}(r_{ext}^4 - r_{int}^4)
\end{equation}

\textbf{Función de costo:}
\begin{equation}
C = \frac{m}{1000} + 1.5\frac{n_{elem}}{50} + 2\frac{n_{nodos}}{100}
\end{equation}

\subsection{Limitaciones del Modelo}

\begin{itemize}
\item Pequeños desplazamientos (lineal geométrico)
\item Comportamiento elástico (sin plasticidad)
\item Elementos tipo barra (sin flexión local)
\item Uniones ideales (conexiones perfectas)
\item Carga estática (sin efectos dinámicos)
\end{itemize}

\section{Métodos}

\subsection{GRUITA 2: Diseño Fijo}

\textbf{Geometría:} Cercha Warren de 30 m con 12 segmentos, altura 1.0 m, torre de 5 m.

\textbf{Elementos:}
\begin{itemize}
\item Montantes: $D_{ext}=50$ mm, $D_{int}=40$ mm (tubular)
\item Cables/diagonales: $D_{ext}=50$ mm, sólidos
\end{itemize}

\textbf{Condiciones de borde:}
\begin{itemize}
\item Torre y base del brazo: fijos ($u_x = u_y = 0$)
\item Carga: 5 N (prueba) a 40 kN (diseño) en punta
\item Peso propio distribuido en todos los nodos
\end{itemize}

\textbf{Análisis de carga móvil:} Variación de carga (0--40 kN) en todas las posiciones del cordón inferior para identificar configuraciones críticas.

\subsection{GRUITA 3: Optimización}

\textbf{Variables de diseño (>20 dimensiones):}
\begin{itemize}
\item Geométricas: $n_{seg} \in [8,20]$, $h_{brazo} \in [0.5,2.0]$ m
\item Topológicas: $tipo_{diag} \in [0,10]$ (11 patrones), espaciamiento vertical
\item Dimensionales: $d_{ext,i} \in [10,50]$ mm, $d_{int,i} \in [5,45]$ mm por elemento
\end{itemize}

\textbf{Patrones topológicos implementados:} Warren, pendiente positiva/negativa, patrón X, abanico inferior/superior, largo alcance, mixto, concentrado, progresivo, completo.

\textbf{Función objetivo con penalizaciones:}
\begin{equation}
\Phi = C + 1000\sum_{i} \max(0, 2-FS_i)^2 + 500\max(0, \delta-0.2)^2
\end{equation}

\textbf{Algoritmo:} Differential Evolution (scipy.optimize)
\begin{itemize}
\item Población: 15, Iteraciones: 100
\item Mutación: $F \in [0.5,1.5]$, Recombinación: 0.7
\item Ventajas: sin gradientes, robusto, explora bien espacios mixtos
\end{itemize}

\section{Resultados}

\subsection{GRUITA 2}

\begin{table}[H]
\small
\centering
\caption{Parámetros GRUITA 2}
\begin{tabular}{lr}
\toprule
\textbf{Parámetro} & \textbf{Valor} \\
\midrule
Nodos & 25 \\
Elementos & 49 \\
Masa & 394.25 kg \\
Costo normalizado & 2.364 \\
\bottomrule
\end{tabular}
\end{table}

\textbf{Desplazamientos (carga 5 N):}
\begin{itemize}
\item Deflexión punta: 1.6 mm
\item Tensión máxima: 0.15 MPa
\item $FS_{tensión} > 1000$, $FS_{pandeo} > 1000$
\end{itemize}

\textbf{Extrapolación a 40 kN:}
\begin{itemize}
\item Deflexión esperada: $\approx 12.8$ m (inaceptable)
\item Esfuerzos: $\approx 1200$ MPa (excede límite elástico)
\item \textbf{Conclusión:} Dimensiones actuales insuficientes para 40 kN
\end{itemize}

\textbf{Análisis de carga móvil:} La posición crítica es el extremo libre (30 m), donde deflexión y esfuerzos son máximos, consistente con teoría de vigas en voladizo.

\subsection{GRUITA 3}

\textbf{Proceso de optimización:}
\begin{enumerate}
\item Inicialización: 15 diseños aleatorios
\item Evaluación: FEM completo por individuo
\item Evolución: mutación y recombinación
\item Selección: diseños mejorantes guardados
\item Convergencia: tolerancia o máximo iteraciones
\end{enumerate}

\begin{table}[H]
\small
\centering
\caption{Diseño optimizado típico}
\begin{tabular}{lr}
\toprule
\textbf{Parámetro} & \textbf{Rango} \\
\midrule
Segmentos & 10--14 \\
Altura brazo & 1.2--1.5 m \\
Patrón & Warren/X \\
Cables soporte & 2--3 \\
\midrule
Masa & 250--350 kg \\
Elementos & 55--75 \\
Costo & 1.5--2.5 \\
\bottomrule
\end{tabular}
\end{table}

\textbf{Ventajas vs diseño manual:}
\begin{itemize}
\item Reducción masa: 10--30\%
\item Distribución eficiente: elementos más gruesos donde hay mayor esfuerzo
\item FS $\geq 2.0$ garantizado en todos los elementos
\item Topología adaptada al patrón de cargas
\end{itemize}

\textbf{Distribución típica de espesores:}
\begin{itemize}
\item Cordones: 40--50 mm (máxima carga axial)
\item Diagonales en apoyos: 30--40 mm
\item Diagonales centrales: 15--25 mm
\item Cables soporte: 45--50 mm (alta tracción)
\end{itemize}

\section{Discusión}

\subsection{Comparación de Enfoques}

\begin{table}[H]
\small
\centering
\caption{GRUITA 2 vs GRUITA 3}
\begin{tabular}{lcc}
\toprule
\textbf{Aspecto} & \textbf{G2} & \textbf{G3} \\
\midrule
Geometría & Fija & Optimizada \\
Topología & Manual & 11 patrones \\
Dimensiones & Uniforme & Individual \\
Costo & Moderado & Mínimo \\
FS & Alta & $\geq 2.0$ \\
Tiempo & Rápido & Horas \\
\bottomrule
\end{tabular}
\end{table}

\subsection{Validación FEM}

Los resultados verifican:
\begin{itemize}
\item Equilibrio: $\sum F = 0$ en nodos libres
\item Compatibilidad: desplazamientos continuos
\item Ley constitutiva: $\sigma = E\epsilon$
\item Condiciones de borde satisfechas
\end{itemize}

\subsection{Mejoras Futuras}

\begin{itemize}
\item Elementos viga (flexión y cortante)
\item Análisis no lineal (grandes desplazamientos)
\item Optimización multi-objetivo
\item Análisis de sensibilidad
\item Modelado de uniones reales
\item Efectos dinámicos y fatiga
\end{itemize}

\section{Conclusiones}

\begin{enumerate}
\item Se implementó exitosamente análisis FEM para grúas reticuladas de 30 m.

\item GRUITA 2 permite validación rápida de conceptos con geometría fija.

\item GRUITA 3 optimiza automáticamente topología y dimensiones, explorando 11 patrones de conectividad.

\item La función de costo balancea peso, complejidad y seguridad efectivamente.

\item Differential Evolution es apropiado para este problema mixto discreto-continuo.

\item El diseño optimizado reduce 10--30\% la masa manteniendo FS $\geq 2.0$.

\item La posición crítica es el extremo libre, consistente con teoría.

\item La metodología es extensible a otras estructuras reticuladas (puentes, torres).

\item Python + NumPy + SciPy proporciona entorno flexible y reproducible.

\item La combinación FEM + optimización evolutiva es poderosa para diseño estructural.
\end{enumerate}

\vspace{0.3cm}

\section*{Referencias}
\small
\begin{itemize}
\item Zienkiewicz, O. C., \& Taylor, R. L. (2000). \textit{The Finite Element Method}. Butterworth-Heinemann.
\item Timoshenko, S. P., \& Gere, J. M. (1961). \textit{Theory of Elastic Stability}. McGraw-Hill.
\item Storn, R., \& Price, K. (1997). Differential evolution. \textit{J. Global Optimization}, 11(4), 341-359.
\item Beer, F. P., et al. (2011). \textit{Mechanics of Materials}. McGraw-Hill.
\end{itemize}

\appendix
\section*{Apéndice: Código Fuente}
\small
\textbf{Archivos disponibles:}
\begin{itemize}
\item \texttt{gruita2.py}: Implementación con geometría fija y análisis de carga móvil
\item \texttt{gruita3.py}: Implementación con optimización automática (Differential Evolution)
\end{itemize}

\textbf{Parámetros materiales:} $E = 200$ GPa, $\rho = 7850$ kg/m$^3$, $\sigma_{adm} = 100$ MPa, $g = 9.81$ m/s$^2$

\textbf{Parámetros de optimización:} Población = 15, Iteraciones = 100, Mutación $F \in [0.5,1.5]$, Recombinación = 0.7

\end{document}
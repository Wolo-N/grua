\documentclass[10pt,a4paper]{article}
\usepackage[spanish]{babel}
\usepackage[utf8]{inputenc}
\usepackage{amsmath}
\usepackage{amssymb}
\usepackage{graphicx}
\usepackage{booktabs}
\usepackage{geometry}
\usepackage{float}
\usepackage{multicol}
\usepackage{hyperref}

\geometry{margin=1.5cm}
\setlength{\parskip}{2pt}
\setlength{\columnsep}{12pt}

\title{\vspace{-1cm}\textbf{Trabajo Práctico 2: Diseño Óptimo de Grúa Torre mediante Elementos Finitos}}
\author{Catalina Dolhare, Nicolas Wolodarsky, Juan Ignacio Castore}


\begin{document}

\maketitle
\vspace{-0.7cm}

\section{Resumen Ejecutivo}

Este trabajo presenta el desarrollo de dos implementaciones computacionales para el análisis y diseño de grúas torre mediante el método de elementos finitos. La primera implementación, \textbf{GRUITA 2}, fue desarrollada con geometría fija inspirada en grúas reales de la industria (modelo ZOOMLION D1500-63) con el objetivo de comprender la mecánica estructural, validar el marco teórico FEM y establecer una base de referencia para futuros desarrollos. La segunda implementación, \textbf{GRUITA 3}, construye sobre esta comprensión e incorpora optimización automática mediante algoritmos evolutivos (Differential Evolution) que exploran simultáneamente topología, geometría y dimensiones para minimizar el costo estructural mientras mantiene factores de seguridad FS $\geq$ 2.0 bajo cargas de diseño de hasta 40 kN. Ambos códigos fueron implementados en Python utilizando NumPy y SciPy, permitiendo análisis reproducible y extensible a otras estructuras reticuladas.

\section{Hipótesis y Marco Teórico}

\subsection{Ecuaciones Fundamentales}

El método de elementos finitos para estructuras reticuladas se basa en las siguientes formulaciones:

\textbf{Matriz de rigidez elemental en coordenadas globales:}
\begin{equation}
\mathbf{K}_e = \frac{EA}{L} \begin{bmatrix}
c^2 & cs & -c^2 & -cs \\
cs & s^2 & -cs & -s^2 \\
-c^2 & -cs & c^2 & cs \\
-cs & -s^2 & cs & s^2
\end{bmatrix}
\end{equation}

donde $E = 200$ GPa (módulo de Young del acero), $A$ es el área de la sección transversal, $L$ es la longitud del elemento, $c = \cos\theta$ y $s = \sin\theta$ definen la orientación del elemento respecto al sistema de coordenadas global.

\textbf{Sistema de ecuaciones global:}
\begin{equation}
\mathbf{K}\mathbf{u} = \mathbf{F}
\end{equation}

donde $\mathbf{K}$ es la matriz de rigidez global ensamblada, $\mathbf{u}$ es el vector de desplazamientos nodales y $\mathbf{F}$ es el vector de fuerzas aplicadas.

\textbf{Esfuerzo axial y deformación:}
\begin{equation}
\sigma = E \frac{\Delta L}{L}, \quad F_{axial} = \sigma \cdot A
\end{equation}

\textbf{Criterio de pandeo de Euler para elementos en compresión:}
\begin{equation}
P_{cr} = \frac{\pi^2 EI}{L^2}, \quad FS_{pandeo} = \frac{P_{cr}}{|P_{compresión}|}
\end{equation}

donde $I$ es el momento de inercia de la sección y $P_{cr}$ es la carga crítica de pandeo.

\textbf{Propiedades geométricas de secciones tubulares circulares huecas:}
\begin{equation}
A = \pi(r_{ext}^2 - r_{int}^2), \quad I = \frac{\pi}{4}(r_{ext}^4 - r_{int}^4)
\end{equation}

\textbf{Función de costo normalizada:}
\begin{equation}
C = \frac{m}{m_0} + 1.5\frac{n_{elem}}{n_{elem,0}} + 2\frac{n_{nodos}}{n_{nodos,0}}
\end{equation}

donde $m_0 = 1000$ kg, $n_{elem,0} = 50$ elementos y $n_{nodos,0} = 100$ nodos son valores de referencia para la normalización. Esta función de costo penaliza la masa, la complejidad estructural (número de elementos) y la complejidad de fabricación (número de uniones).

\subsection{Limitaciones del Modelo}

El modelo adoptado presenta las siguientes simplificaciones que definen su alcance y validez: se asume régimen de pequeños desplazamientos (linealidad geométrica), comportamiento elástico lineal del material sin considerar plasticidad o fractura, elementos tipo barra que transmiten únicamente fuerzas axiales sin capacidad de flexión local, uniones articuladas ideales sin fricción ni holguras, y cargas estáticas sin efectos dinámicos, de impacto o de fatiga. Estas hipótesis son razonables para un análisis preliminar de diseño y proporcionan una aproximación conservadora del comportamiento estructural real.

\section{GRUITA 2: Fase de Comprensión y Validación}

Para comprender la mecánica de grúas torre y validar el marco teórico FEM, se desarrolló \textbf{GRUITA 2} con geometría fija inspirada en el modelo real ZOOMLION D1500-63, una grúa torre industrial con alcance máximo de 70 m y capacidad de carga variable (21--63 t según configuración). Basándonos en las especificaciones técnicas de esta grúa, que presenta una estructura reticulada tipo Warren con altura de torre de 85 m, brazo horizontal segmentado y sistema de cables de soporte, diseñamos un modelo simplificado a escala académica con las siguientes características:

\textbf{Geometría del brazo:} Cercha Warren de 30 m de largo, dividida en 12 segmentos de 2.5 m cada uno, con altura de perfil de 1.0 m. Esta topología Warren (diagonales alternadas) es estándar en la industria por su eficiencia estructural y simplicidad de fabricación.

\textbf{Torre de soporte:} Torre vertical de 5 m de altura con base empotrada, representando el mástil principal de la grúa.

\textbf{Cables de soporte:} Dos cables diagonales conectan la punta del brazo con el tope de la torre, simulando el sistema de tirantes presente en grúas reales que contrarresta momentos de flexión.

\textbf{Dimensiones de secciones:}
\begin{itemize}
\item Montantes (cordones superior e inferior): tubos circulares $D_{ext}=50$ mm, $D_{int}=40$ mm
\item Diagonales: tubos circulares $D_{ext}=50$ mm, $D_{int}=40$ mm
\item Cables de soporte: barras sólidas $D=50$ mm
\end{itemize}

\textbf{Materiales:} Acero estructural con $E = 200$ GPa, $\rho = 7850$ kg/m$^3$, $\sigma_{adm} = 100$ MPa.

\textbf{Condiciones de contorno:}
\begin{itemize}
\item Base de la torre: empotrada ($u_x = u_y = 0$)
\item Base del brazo (conexión con torre): empotrada ($u_x = u_y = 0$)
\item Carga vertical concentrada en la punta del brazo: 5 N (prueba) hasta 40 kN (diseño)
\item Peso propio distribuido en todos los nodos según masa de elementos conectados
\end{itemize}

\textbf{Análisis de carga móvil:} Se implementó análisis paramétrico donde la carga se desplaza a lo largo del cordón inferior del brazo (posiciones de 0 a 30 m) para identificar configuraciones críticas de deflexión y esfuerzos, simulando el movimiento del carro portacarga en grúas reales.

\textbf{Objetivos cumplidos:}
\begin{enumerate}
\item Validar implementación del método FEM (equilibrio, compatibilidad, ley constitutiva)
\item Verificar cálculo de factores de seguridad (tensión y pandeo)
\item Comprender distribución de esfuerzos en topología Warren
\item Identificar posiciones críticas de carga
\item Establecer marco de referencia para optimización posterior
\end{enumerate}

\textbf{Resultados de GRUITA 2:}

\begin{table}[H]
\small
\centering
\caption{Características estructurales de GRUITA 2}
\begin{tabular}{lr}
\toprule
\textbf{Parámetro} & \textbf{Valor} \\
\midrule
Número de nodos & 25 \\
Número de elementos & 49 \\
Masa total & 394.25 kg \\
Costo normalizado & 2.364 \\
\bottomrule
\end{tabular}
\end{table}

Con carga de prueba de 5 N en la punta, se observó deflexión vertical de 1.6 mm, esfuerzos máximos de 0.15 MPa y factores de seguridad superiores a 1000 (estructura altamente sobredimensionada). La extrapolación lineal a 40 kN sugiere deflexión de aproximadamente 12.8 m y esfuerzos de 1200 MPa, ambos valores físicamente inaceptables que confirman la necesidad de redimensionar la estructura para cargas de diseño reales. El análisis de carga móvil confirmó que la posición crítica es el extremo libre (30 m), consistente con la teoría de vigas en voladizo.

\section{GRUITA 3: Fase de Optimización Automática}

Una vez validado el marco teórico con GRUITA 2, se procedió a desarrollar \textbf{GRUITA 3}, una implementación que genera y optimiza automáticamente la grúa mediante algoritmos evolutivos. El código explora simultáneamente tres dimensiones del espacio de diseño: topología (patrones de conectividad), geometría (longitudes y alturas), y dimensiones (diámetros de secciones transversales).

\subsection{Variables de Diseño}

El problema de optimización involucra más de 20 variables continuas y discretas, organizadas en tres categorías:

\begin{table}[H]
\small
\centering
\caption{Variables de diseño en GRUITA 3}
\begin{tabular}{llcc}
\toprule
\textbf{Categoría} & \textbf{Variable} & \textbf{Rango} & \textbf{Tipo} \\
\midrule
\multirow{3}{*}{\textbf{Geométricos}}
& Número de segmentos del brazo & $n_{seg} \in [8, 20]$ & Entero \\
& Altura del perfil del brazo & $h_{brazo} \in [0.5, 2.0]$ m & Continuo \\
& Altura de la torre & 5 m & Fijo \\
\midrule
\multirow{3}{*}{\textbf{Topológicos}}
& Tipo de patrón diagonal & $tipo \in [0, 10]$ & Discreto (11 opciones) \\
& Número de cables de soporte & $n_{cables} \in [0, 3]$ & Entero \\
& Espaciamiento vertical de cables & Variable & Continuo \\
\midrule
\multirow{2}{*}{\textbf{Dimensionales}}
& Diámetro externo (por elemento) & $d_{ext,i} \in [10, 50]$ mm & Continuo \\
& Diámetro interno (por elemento) & $d_{int,i} \in [5, d_{ext,i}-5]$ mm & Continuo \\
\bottomrule
\end{tabular}
\end{table}

\subsection{Patrones Topológicos Implementados}

El código \texttt{design\_parametric\_crane()} genera 11 patrones diferentes de conectividad:
\begin{enumerate}
\item \textbf{Warren clásico:} Diagonales alternadas en zig-zag
\item \textbf{Pendiente positiva:} Todas las diagonales con inclinación $+45°$
\item \textbf{Pendiente negativa:} Todas las diagonales con inclinación $-45°$
\item \textbf{Patrón X:} Diagonales cruzadas en cada panel
\item \textbf{Abanico inferior:} Diagonales conectadas al cordón inferior
\item \textbf{Abanico superior:} Diagonales conectadas al cordón superior
\item \textbf{Largo alcance:} Diagonales que saltan un panel
\item \textbf{Mixto:} Combinación Warren + X en regiones alternadas
\item \textbf{Concentrado:} Mayor densidad de diagonales cerca de la base
\item \textbf{Progresivo:} Densidad variable a lo largo del brazo
\item \textbf{Completo:} Máxima conectividad (todos los nodos conectados)
\end{enumerate}

\subsection{Función Objetivo con Penalizaciones}

La función de evaluación \texttt{evaluate\_crane\_design()} combina el costo normalizado con términos de penalización cuadrática para garantizar factibilidad estructural:

\begin{equation}
\Phi = C(m, n_{elem}, n_{nodos}) + \alpha \sum_{i=1}^{n_{elem}} \max(0, 2-FS_i)^2 + \beta \max(0, \delta_{max}-\delta_{lim})^2
\end{equation}

donde:
\begin{itemize}
\item $C$ es el costo normalizado definido anteriormente
\item $\alpha = 1000$ es el coeficiente de penalización por factores de seguridad inadecuados
\item $FS_i$ es el factor de seguridad (tensión o pandeo) del elemento $i$
\item $\beta = 500$ es el coeficiente de penalización por deflexión excesiva
\item $\delta_{max}$ es la deflexión máxima observada, $\delta_{lim} = 0.2$ m
\end{itemize}

Las penalizaciones cuadráticas garantizan que diseños no factibles (con $FS < 2.0$ o deflexiones excesivas) reciban valores de función objetivo muy altos, forzando al algoritmo a explorar regiones factibles del espacio de diseño.

\subsection{Algoritmo de Optimización}

Se utiliza \textbf{Differential Evolution} (DE) de la librería \texttt{scipy.optimize}, un algoritmo evolutivo robusto para problemas no lineales, no convexos, con variables mixtas (continuas y discretas). DE es particularmente apropiado para este problema porque no requiere gradientes (el espacio de diseño es no diferenciable debido a las variables discretas) y explora eficientemente espacios de alta dimensión mediante mutación y recombinación.

\textbf{Parámetros del algoritmo:}
\begin{itemize}
\item Tamaño de población: 15 individuos
\item Número de generaciones: 100 iteraciones
\item Factor de mutación: $F \in [0.5, 1.5]$ (dithering para evitar convergencia prematura)
\item Tasa de recombinación: $CR = 0.7$
\item Estrategia: \texttt{best1bin} (mutación basada en el mejor individuo actual)
\end{itemize}

\textbf{Proceso de optimización:}
\begin{enumerate}
\item \textbf{Inicialización:} Generar 15 diseños aleatorios dentro de los límites especificados
\item \textbf{Evaluación:} Para cada diseño, ejecutar análisis FEM completo y calcular $\Phi$
\item \textbf{Mutación:} Crear vectores mutantes mediante $\mathbf{v}_i = \mathbf{x}_{best} + F(\mathbf{x}_{r1} - \mathbf{x}_{r2})$
\item \textbf{Recombinación:} Combinar individuo original con mutante según probabilidad $CR$
\item \textbf{Selección:} Reemplazar individuo solo si el nuevo diseño mejora $\Phi$
\item \textbf{Convergencia:} Detener cuando se alcanza tolerancia o número máximo de iteraciones
\end{enumerate}

Cada evaluación de función objetivo requiere resolver el sistema FEM completo (ensamblaje de matriz de rigidez global, aplicación de condiciones de contorno, resolución del sistema lineal, cálculo de fuerzas axiales, verificación de tensión y pandeo), por lo que el tiempo computacional total es del orden de horas dependiendo de la complejidad de los diseños explorados.

\section{Resultados}

\subsection{Resultados de GRUITA 3}

\textit{[Espacio reservado para resultados una vez finalizada la optimización]}

\vspace{1cm}

\begin{table}[H]
\small
\centering
\caption{Diseño óptimo obtenido (a completar)}
\begin{tabular}{lr}
\toprule
\textbf{Parámetro} & \textbf{Valor} \\
\midrule
Número de segmentos & \\
Altura del brazo (m) & \\
Patrón topológico & \\
Número de cables soporte & \\
\midrule
Masa total (kg) & \\
Número de elementos & \\
Número de nodos & \\
Costo normalizado & \\
\midrule
FS mínimo (tensión) & \\
FS mínimo (pandeo) & \\
Deflexión máxima (m) & \\
\bottomrule
\end{tabular}
\end{table}

\vspace{0.5cm}

\textbf{Análisis de convergencia:}

\textit{[Gráfico de evolución de función objetivo vs. generación]}

\textit{[Descripción de velocidad de convergencia, número de evaluaciones, diseños no factibles descartados]}

\vspace{0.5cm}

\textbf{Distribución de esfuerzos:}

\textit{[Tabla o gráfico mostrando factores de seguridad por elemento]}

\textit{[Descripción de elementos críticos, distribución de tensiones, elementos en tracción vs. compresión]}

\vspace{0.5cm}

\textbf{Análisis de sensibilidad:}

\textit{[Discusión sobre robustez del diseño óptimo, sensibilidad a variaciones de parámetros]}

\section{Conclusión}

Este trabajo demuestra la viabilidad de combinar el método de elementos finitos con algoritmos de optimización evolutiva para el diseño automático de estructuras reticuladas complejas. La estrategia de desarrollo en dos fases resultó fundamental: \textbf{GRUITA 2} permitió comprender la mecánica estructural, validar el marco teórico FEM (equilibrio de fuerzas, compatibilidad de desplazamientos, relación constitutiva) y establecer una referencia basada en grúas reales de la industria, mientras que \textbf{GRUITA 3} construyó sobre esta base para explorar automáticamente un espacio de diseño de alta dimensión (topología, geometría, dimensiones) mediante Differential Evolution.

El enfoque de penalización cuadrática para garantizar factibilidad estructural (factores de seguridad FS $\geq 2.0$, deflexiones admisibles) demostró ser efectivo para guiar el algoritmo hacia regiones factibles del espacio de diseño. La inclusión de 11 patrones topológicos distintos permite al optimizador seleccionar conectividades que balancean eficiencia estructural con complejidad de fabricación, reflejando trade-offs presentes en la ingeniería real.

La función de costo multi-objetivo, que combina masa, número de elementos y número de uniones, captura aspectos económicos relevantes: el costo del material, la complejidad de fabricación y el costo de ensamblaje en campo. Los pesos relativos (1.0, 1.5, 2.0) reflejan que las uniones son típicamente más costosas que los elementos adicionales, y estos más costosos que el material puro.

Python con NumPy y SciPy resultó ser un entorno apropiado para este problema, proporcionando álgebra lineal eficiente, algoritmos de optimización robustos y capacidad de visualización para análisis post-procesamiento. El código es modular, documentado y extensible a otros tipos de estructuras reticuladas (puentes, torres de transmisión, techos espaciales).

\vspace{0.5cm}

\textit{[Espacio reservado para conclusiones adicionales basadas en resultados numéricos finales de GRUITA 3]}

\vspace{0.3cm}

\textbf{Limitaciones y trabajo futuro:}

El modelo actual asume elementos tipo barra (solo fuerzas axiales) y pequeños desplazamientos, simplificaciones razonables para diseño preliminar pero que podrían extenderse a elementos viga con capacidad de flexión y cortante, análisis no lineal geométrico para grandes desplazamientos, consideración de plasticidad del material, modelado de uniones reales con rigideces finitas, análisis dinámico y de fatiga, y optimización multi-objetivo explícita (Pareto fronts para trade-offs masa vs. complejidad vs. robustez).

\section*{Referencias}
\small
\begin{itemize}
\item Zienkiewicz, O. C., \& Taylor, R. L. (2000). \textit{The Finite Element Method: Volume 1, The Basis}. Butterworth-Heinemann.
\item Timoshenko, S. P., \& Gere, J. M. (1961). \textit{Theory of Elastic Stability}. McGraw-Hill.
\item Storn, R., \& Price, K. (1997). Differential evolution – A simple and efficient heuristic for global optimization over continuous spaces. \textit{Journal of Global Optimization}, 11(4), 341-359.
\item Beer, F. P., Johnston, E. R., DeWolf, J. T., \& Mazurek, D. F. (2011). \textit{Mechanics of Materials} (6th ed.). McGraw-Hill.
\item ZOOMLION (2012). \textit{D1500-63 Tower Crane Technical Specifications}. Manual técnico.
\end{itemize}

\appendix
\section*{Apéndice: Implementación Computacional}

\textbf{Archivos del proyecto:}
\begin{itemize}
\item \texttt{gruita2.py}: Implementación con geometría fija basada en grúa real ZOOMLION D1500-63. Incluye análisis FEM, cálculo de factores de seguridad, análisis de carga móvil y funciones de visualización.
\item \texttt{gruita3.py}: Implementación con optimización automática mediante Differential Evolution. Incluye generador paramétrico de topologías, evaluador FEM con penalizaciones y rutina de optimización.
\end{itemize}

\textbf{Parámetros materiales:} $E = 200$ GPa, $\rho = 7850$ kg/m$^3$, $\sigma_{adm} = 100$ MPa, $g = 9.81$ m/s$^2$

\textbf{Parámetros de optimización:} Población = 15, Generaciones = 100, Mutación $F \in [0.5, 1.5]$, Recombinación $CR = 0.7$

\end{document}
